\documentclass[12pt,fleqn,leqno,letterpaper]{article}

\title{Distributed Systems Assignment}
\author{Ryan Collins\\
  \small{Username - gcdk35}\\
}
\date{\today}

\begin{document}

\maketitle

\section*{Question 1}
\subsection*{Part A}
When communicating with a system employing passive replication, the following timeline is followed:

\paragraph{Request sent from client to front-end}
When the request is received by the front-end, if the request has previously been received, it's merely ignored, with the previous response being echoed (as the state has already been changed). Otherwise, this request is sent to the primary server.

\paragraph{If: the primary server isn't available}
In the event no response is received from the primary server after a given time, we assume the server to be offline.
Therefore, we remove that server from being the primary and set one of the backup servers as the primary server.

\paragraph{Otherwise: Request processed by primary server}
    The request is processed, and depending on the action, this is carried out. If there has been a change in state of the
    primary server (i.e. data has been created, updated or deleted), then this state is propagated through to both backup
    servers. If we are just retrieving information, there is no need to waste resources on updating a state that hasn't
    changed.

\subsection*{Part B}
Passive replication ensures that several different copies of the data exist. If for some reason data is lost in one location, the data from another location can simply be copied
from another server, in the replication process. If all but one of the servers go down (in this case, the order manager servers), the data will still be kept intact.

\section*{Question 2}
    The code and working system is provided in the project. Just run the respective files to test the system.

\section*{Question 3}
\subsection*{Part A}
    When the primary server fails, it should be removed from the pool of working servers, and one of the backup servers
    should take over. If the server begins working again, then it should be able to be added back into the pool of servers as a backup server.

\subsection*{Part B}
Rather than selecting the remote server from the self.server variable, we instead run a function to pick the primary server.
 This function simply iterates through each server, one by one, and if a connection error occurs, the server itself is ignored
 and we move onto the next one. We then set the first working server to be the primary server, and the remaining servers
 as the non-primary servers. As there are three servers, imagine OrderManager1 goes offline. Then, OrderManager2 would be
 made the primary server and OrderManager3 would remain the backup server (sadly the only backup server). If OrderManager1
 comes back online, it will once again replace OrderManager2 as the primary server, and OrderManager2 will revert to being
 a backup server.

The code can be found in the Part2 folder, which contains the working program.
\end{document}